% Options for packages loaded elsewhere
\PassOptionsToPackage{unicode}{hyperref}
\PassOptionsToPackage{hyphens}{url}
%
\documentclass[
]{article}
\usepackage{amsmath,amssymb}
\usepackage{iftex}
\ifPDFTeX
  \usepackage[T1]{fontenc}
  \usepackage[utf8]{inputenc}
  \usepackage{textcomp} % provide euro and other symbols
\else % if luatex or xetex
  \usepackage{unicode-math} % this also loads fontspec
  \defaultfontfeatures{Scale=MatchLowercase}
  \defaultfontfeatures[\rmfamily]{Ligatures=TeX,Scale=1}
\fi
\usepackage{lmodern}
\ifPDFTeX\else
  % xetex/luatex font selection
\fi
% Use upquote if available, for straight quotes in verbatim environments
\IfFileExists{upquote.sty}{\usepackage{upquote}}{}
\IfFileExists{microtype.sty}{% use microtype if available
  \usepackage[]{microtype}
  \UseMicrotypeSet[protrusion]{basicmath} % disable protrusion for tt fonts
}{}
\makeatletter
\@ifundefined{KOMAClassName}{% if non-KOMA class
  \IfFileExists{parskip.sty}{%
    \usepackage{parskip}
  }{% else
    \setlength{\parindent}{0pt}
    \setlength{\parskip}{6pt plus 2pt minus 1pt}}
}{% if KOMA class
  \KOMAoptions{parskip=half}}
\makeatother
\usepackage{xcolor}
\usepackage[margin=1in]{geometry}
\usepackage{color}
\usepackage{fancyvrb}
\newcommand{\VerbBar}{|}
\newcommand{\VERB}{\Verb[commandchars=\\\{\}]}
\DefineVerbatimEnvironment{Highlighting}{Verbatim}{commandchars=\\\{\}}
% Add ',fontsize=\small' for more characters per line
\usepackage{framed}
\definecolor{shadecolor}{RGB}{248,248,248}
\newenvironment{Shaded}{\begin{snugshade}}{\end{snugshade}}
\newcommand{\AlertTok}[1]{\textcolor[rgb]{0.94,0.16,0.16}{#1}}
\newcommand{\AnnotationTok}[1]{\textcolor[rgb]{0.56,0.35,0.01}{\textbf{\textit{#1}}}}
\newcommand{\AttributeTok}[1]{\textcolor[rgb]{0.13,0.29,0.53}{#1}}
\newcommand{\BaseNTok}[1]{\textcolor[rgb]{0.00,0.00,0.81}{#1}}
\newcommand{\BuiltInTok}[1]{#1}
\newcommand{\CharTok}[1]{\textcolor[rgb]{0.31,0.60,0.02}{#1}}
\newcommand{\CommentTok}[1]{\textcolor[rgb]{0.56,0.35,0.01}{\textit{#1}}}
\newcommand{\CommentVarTok}[1]{\textcolor[rgb]{0.56,0.35,0.01}{\textbf{\textit{#1}}}}
\newcommand{\ConstantTok}[1]{\textcolor[rgb]{0.56,0.35,0.01}{#1}}
\newcommand{\ControlFlowTok}[1]{\textcolor[rgb]{0.13,0.29,0.53}{\textbf{#1}}}
\newcommand{\DataTypeTok}[1]{\textcolor[rgb]{0.13,0.29,0.53}{#1}}
\newcommand{\DecValTok}[1]{\textcolor[rgb]{0.00,0.00,0.81}{#1}}
\newcommand{\DocumentationTok}[1]{\textcolor[rgb]{0.56,0.35,0.01}{\textbf{\textit{#1}}}}
\newcommand{\ErrorTok}[1]{\textcolor[rgb]{0.64,0.00,0.00}{\textbf{#1}}}
\newcommand{\ExtensionTok}[1]{#1}
\newcommand{\FloatTok}[1]{\textcolor[rgb]{0.00,0.00,0.81}{#1}}
\newcommand{\FunctionTok}[1]{\textcolor[rgb]{0.13,0.29,0.53}{\textbf{#1}}}
\newcommand{\ImportTok}[1]{#1}
\newcommand{\InformationTok}[1]{\textcolor[rgb]{0.56,0.35,0.01}{\textbf{\textit{#1}}}}
\newcommand{\KeywordTok}[1]{\textcolor[rgb]{0.13,0.29,0.53}{\textbf{#1}}}
\newcommand{\NormalTok}[1]{#1}
\newcommand{\OperatorTok}[1]{\textcolor[rgb]{0.81,0.36,0.00}{\textbf{#1}}}
\newcommand{\OtherTok}[1]{\textcolor[rgb]{0.56,0.35,0.01}{#1}}
\newcommand{\PreprocessorTok}[1]{\textcolor[rgb]{0.56,0.35,0.01}{\textit{#1}}}
\newcommand{\RegionMarkerTok}[1]{#1}
\newcommand{\SpecialCharTok}[1]{\textcolor[rgb]{0.81,0.36,0.00}{\textbf{#1}}}
\newcommand{\SpecialStringTok}[1]{\textcolor[rgb]{0.31,0.60,0.02}{#1}}
\newcommand{\StringTok}[1]{\textcolor[rgb]{0.31,0.60,0.02}{#1}}
\newcommand{\VariableTok}[1]{\textcolor[rgb]{0.00,0.00,0.00}{#1}}
\newcommand{\VerbatimStringTok}[1]{\textcolor[rgb]{0.31,0.60,0.02}{#1}}
\newcommand{\WarningTok}[1]{\textcolor[rgb]{0.56,0.35,0.01}{\textbf{\textit{#1}}}}
\usepackage{longtable,booktabs,array}
\usepackage{calc} % for calculating minipage widths
% Correct order of tables after \paragraph or \subparagraph
\usepackage{etoolbox}
\makeatletter
\patchcmd\longtable{\par}{\if@noskipsec\mbox{}\fi\par}{}{}
\makeatother
% Allow footnotes in longtable head/foot
\IfFileExists{footnotehyper.sty}{\usepackage{footnotehyper}}{\usepackage{footnote}}
\makesavenoteenv{longtable}
\usepackage{graphicx}
\makeatletter
\def\maxwidth{\ifdim\Gin@nat@width>\linewidth\linewidth\else\Gin@nat@width\fi}
\def\maxheight{\ifdim\Gin@nat@height>\textheight\textheight\else\Gin@nat@height\fi}
\makeatother
% Scale images if necessary, so that they will not overflow the page
% margins by default, and it is still possible to overwrite the defaults
% using explicit options in \includegraphics[width, height, ...]{}
\setkeys{Gin}{width=\maxwidth,height=\maxheight,keepaspectratio}
% Set default figure placement to htbp
\makeatletter
\def\fps@figure{htbp}
\makeatother
\setlength{\emergencystretch}{3em} % prevent overfull lines
\providecommand{\tightlist}{%
  \setlength{\itemsep}{0pt}\setlength{\parskip}{0pt}}
\setcounter{secnumdepth}{-\maxdimen} % remove section numbering
\ifLuaTeX
  \usepackage{selnolig}  % disable illegal ligatures
\fi
\IfFileExists{bookmark.sty}{\usepackage{bookmark}}{\usepackage{hyperref}}
\IfFileExists{xurl.sty}{\usepackage{xurl}}{} % add URL line breaks if available
\urlstyle{same}
\hypersetup{
  hidelinks,
  pdfcreator={LaTeX via pandoc}}

\author{}
\date{\vspace{-2.5em}}

\begin{document}

\hypertarget{aufgabenblatt-9}{%
\section{Aufgabenblatt 9}\label{aufgabenblatt-9}}

\hypertarget{statistik-fuxfcr-wirtschaftsinformatiker-uxfcbung-htw-berlin}{%
\subsection{Statistik für Wirtschaftsinformatiker, Übung, HTW
Berlin}\label{statistik-fuxfcr-wirtschaftsinformatiker-uxfcbung-htw-berlin}}

\hypertarget{michael-heimann-shirin-riazy}{%
\subsubsection{Michael Heimann, Shirin
Riazy}\label{michael-heimann-shirin-riazy}}

Stand: 14.06.2023

\hypertarget{wiederholung}{%
\subsection{Wiederholung}\label{wiederholung}}

\begin{itemize}
\tightlist
\item
  Was ist Korrelation?
\item
  Was ist lineare Regression?
\item
  Was ist die Gleichung für eine Gerade?
\item
  Welche ist die unabhängige und welche ist die abhängige Variable in
  der Gleichung?
\end{itemize}

\hypertarget{aufgabe-9.1-regressionsrechnung}{%
\subsection{Aufgabe 9.1
(Regressionsrechnung)}\label{aufgabe-9.1-regressionsrechnung}}

Laden Sie den Datensatz \texttt{Readiq} aus dem R-Paket \texttt{BSDA}.

\begin{Shaded}
\begin{Highlighting}[]
\FunctionTok{library}\NormalTok{(BSDA)}
\FunctionTok{data}\NormalTok{(Readiq)}
\end{Highlighting}
\end{Shaded}

\begin{enumerate}
\def\labelenumi{\alph{enumi})}
\tightlist
\item
  Machen Sie sich mit dem Datensatz vertraut. Was sind die beiden
  Merkmale in dem Datensatz?
\item
  Fertigen Sie ein Streudiagramm (Scatterplot) an.
\item
  Bestimmem Sie den Korrelationskoeffizienten nach Pearson für die
  beiden Merkmale. Liegt keine, eine schwache, eine mittlere oder eine
  starke positive bzw. negative Korrelation zwischen beiden Merkmalen
  vor? Was bedeutet dies?
\item
  Bestimmen Sie die Geradengleichung der linearen Regression mit
  \texttt{Readiq\$reading} als unabhängige und \texttt{Readiq\$iq} als
  abhängige Variable. Berechnen Sie dazu die Werte von \(a\) und \(b\)
  in R mit den Formeln aus der Vorlesung. Hinweis: Obwohl \texttt{sd()}
  in R die Stichprobenvarianz und nicht die empirische Varianz aus den
  Formeln ist, kann \texttt{sd()} benutzt werden, da sich die Faktoren
  (\(1/n\) bzw. \(1/(n+1)\)) in der Formel herauskürzen.
\item
  Fügen Sie dem Streuungsdiagramm von \texttt{Readiq\$reading} und
  \texttt{Readiq\$iq} die Regressionsgerade zu. Hinweis: Benutze
  \texttt{abline()}.
\end{enumerate}

\begin{Shaded}
\begin{Highlighting}[]
\CommentTok{\#a}
\FunctionTok{View}\NormalTok{(Readiq)}
\CommentTok{\# Merkmale sind reading und iq}

\CommentTok{\#b}
\FunctionTok{scatter.smooth}\NormalTok{(Readiq)}
\end{Highlighting}
\end{Shaded}

\includegraphics{aufgaben_sta_9_files/figure-latex/unnamed-chunk-2-1.pdf}

\begin{Shaded}
\begin{Highlighting}[]
\CommentTok{\#c}
\FunctionTok{library}\NormalTok{(DescTools)}
\FunctionTok{ContCoef}\NormalTok{(Readiq)}
\end{Highlighting}
\end{Shaded}

\begin{verbatim}
## [1] 0.08921001
\end{verbatim}

\begin{Shaded}
\begin{Highlighting}[]
\CommentTok{\# Korrelationskoeffizient von 0.08, sehr niedrig, schwache positive Korrelation, heißt kein Zusammenhang zwischen IQ und Lesewerten}
\end{Highlighting}
\end{Shaded}

\hypertarget{aufgabe-9.2-regressionsrechnung}{%
\subsection{Aufgabe 9.2
(Regressionsrechnung)}\label{aufgabe-9.2-regressionsrechnung}}

Über die Einführung des Mindestlohns wurde lange diskutiert. Ende 2007
veröffentlichte die \emph{Wirtschaftswoche} folgende Daten zu
Mindestlohn (in Euro) und Arbeitslosenquote (in Prozent):

\begin{Shaded}
\begin{Highlighting}[]
\FunctionTok{library}\NormalTok{(knitr)}
\NormalTok{mindestlohn\_tabelle }\OtherTok{\textless{}{-}} \FunctionTok{data.frame}\NormalTok{(}\AttributeTok{row.names =} \FunctionTok{c}\NormalTok{(}\StringTok{"Irland"}\NormalTok{, }\StringTok{"Frankreich"}\NormalTok{, }\StringTok{"Großbritannien"}\NormalTok{, }
                                            \StringTok{"Belgien"}\NormalTok{, }\StringTok{"Niederlande"}\NormalTok{, }\StringTok{"USA"}\NormalTok{, }\StringTok{"Spanien"}\NormalTok{), }
                          \AttributeTok{Mindestlohn =} \FunctionTok{c}\NormalTok{(}\FloatTok{8.65}\NormalTok{, }\FloatTok{4.44}\NormalTok{, }\FloatTok{8.2}\NormalTok{, }\FloatTok{8.08}\NormalTok{, }\FloatTok{8.08}\NormalTok{, }\FloatTok{4.3}\NormalTok{, }\FloatTok{3.42}\NormalTok{), }
                          \AttributeTok{Arbeitslosenquote =} \FunctionTok{c}\NormalTok{(}\FloatTok{4.4}\NormalTok{, }\FloatTok{9.0}\NormalTok{, }\FloatTok{5.5}\NormalTok{, }\FloatTok{8.2}\NormalTok{, }\FloatTok{5.5}\NormalTok{, }\FloatTok{4.6}\NormalTok{, }\FloatTok{8.5}\NormalTok{))}

\FunctionTok{kable}\NormalTok{(mindestlohn\_tabelle)}
\end{Highlighting}
\end{Shaded}

\begin{longtable}[]{@{}lrr@{}}
\toprule\noalign{}
& Mindestlohn & Arbeitslosenquote \\
\midrule\noalign{}
\endhead
\bottomrule\noalign{}
\endlastfoot
Irland & 8.65 & 4.4 \\
Frankreich & 4.44 & 9.0 \\
Großbritannien & 8.20 & 5.5 \\
Belgien & 8.08 & 8.2 \\
Niederlande & 8.08 & 5.5 \\
USA & 4.30 & 4.6 \\
Spanien & 3.42 & 8.5 \\
\end{longtable}

Gibt es einen linearen Zusammenhang zwischen \emph{Mindestlohn}
(unabhängige Variable) und \emph{Arbeitslosenquote}? Stellen Sie die
Daten in einem Streudiagramm dar, berechnen Sie die Regressionsgerade
und fügen Sie sie dem Streudiagramm zu. Ist die Regressionsgerade eine
sinnvolle Beschreibung der Daten?

Hinweis: Benutzen Sie diesmal die Funktion \texttt{lm()} in R zur
Berechnung der Regressionsgeraden. Weisen Sie das Ergebnis des Aufrufs
einer Variablen zu: \texttt{regression\ \textless{}-\ lm(...)}.
Untersuchen Sie die Ausgabe von \texttt{regression} und
\texttt{summary(regression)} und vergleichen Sie zur Interpretation auch
die Folien der Vorlesung.

\hypertarget{aufgabe-9.3-regressionsrechnung}{%
\subsection{Aufgabe 9.3
(Regressionsrechnung)}\label{aufgabe-9.3-regressionsrechnung}}

Benutzen Sie wieder die Daten des Mietspiegels in München von 2003.

\begin{Shaded}
\begin{Highlighting}[]
\NormalTok{mietspiegel }\OtherTok{\textless{}{-}} \FunctionTok{read.table}\NormalTok{(}\StringTok{"../miete03.asc"}\NormalTok{, }\AttributeTok{header=}\ConstantTok{TRUE}\NormalTok{)}
\end{Highlighting}
\end{Shaded}

\begin{enumerate}
\def\labelenumi{\alph{enumi})}
\tightlist
\item
  Stellen Sie den Zusammenhang zwischen der Wohnfläche
  \texttt{mietspiegel\$wfl} als unabhängiger und Nettomiete
  \texttt{mietspiegel\$nm} als abhängiger Variable als Streudiagramm und
  durch die Regressionsgerade dar.
\item
  Stellen Sie die Geradengleichung auf und berechnen die Nettomieten für
  die Wohnflächen 60qm, 100qm und 150qm, die sich aus der
  Geradengleichung als Schätzwerte ergeben.
\item
  Betrachten Sie wie in Aufgabe 9.2 das Regressionsobjekt mit
  \texttt{summary()}. Wie viel Prozent der Varianz von Nettomiete wird
  durch die Wohnfläche erklärt?
\end{enumerate}

\hypertarget{aufgabe-9.4-mehrdimensionale-regressionsrechnung}{%
\subsection{Aufgabe 9.4 (Mehrdimensionale
Regressionsrechnung)}\label{aufgabe-9.4-mehrdimensionale-regressionsrechnung}}

Wir benutzen \texttt{mietspiegel} aus Aufgabe 9.3 und untersuchen, ob
wir Nettomiete besser schätzen können, indem wir weitere Informationen
berücksichtigen. Wir können z.B. das Merkmal \texttt{wohnbest} durch

\begin{Shaded}
\begin{Highlighting}[]
\FunctionTok{lm}\NormalTok{(nm }\SpecialCharTok{\textasciitilde{}}\NormalTok{ wfl }\SpecialCharTok{+}\NormalTok{ wohnbest, }\AttributeTok{data =}\NormalTok{ mietspiegel)}
\end{Highlighting}
\end{Shaded}

zusätzlich zu \texttt{wfl} als Information der Regression zufügen.

Die Qualität des Regressionsmodells kann in \texttt{summary()} anhand
der Werte von

\begin{itemize}
\tightlist
\item
  \emph{Residuals} (je kleiner die Beträge, desto besser)
\item
  \emph{Residual Standard Error} (je kleiner desto besser)
\item
  \emph{Adjusted R-Squared} (je größer desto besser)
\end{itemize}

abgelesen werden.

Fügen Sie \texttt{wohnbest}, \texttt{ww0} und \texttt{kueche} jeweils
einzeln und in Kombination dem Regressionsmodell zu und vergleichen Sie
die Qualität der Modelle mit Hilfe von \texttt{summary()}. Welches ist
das beste Modell?

\hypertarget{aufgabe-9.5-regressionsrechnung-mit-polynomen-huxf6herer-ordnung}{%
\subsection{Aufgabe 9.5 (Regressionsrechnung mit Polynomen höherer
Ordnung)}\label{aufgabe-9.5-regressionsrechnung-mit-polynomen-huxf6herer-ordnung}}

Lineare Regression kann auch mit komplexeren Funktionen durchgeführt
werden wie z.B. quadratischen Funktionen
\(f(x) = a_1 \cdot x + a_2 \cdot x^2 + b\). In R kann man für diesen
Fall statt der Formel \texttt{y\ \textasciitilde{}\ x} in der Funktion
\texttt{lm()} Formeln wie \texttt{y\ \textasciitilde{}\ poly(x,2)} für
ein quadratisches Polynom benutzen.

Berechnen Sie für den folgenden Datensatz \((x,y)\) eine lineare
Regression mit linearen, quadratischen und kubischen Formeln. Plotten
Sie alle drei Ergebnisse als Kurven (Linienplot) im Streudiagramm der
Daten. Vergleichen Sie die Güte der drei Modelle über
\texttt{summary()}.

\begin{Shaded}
\begin{Highlighting}[]
\NormalTok{x }\OtherTok{\textless{}{-}} \FunctionTok{runif}\NormalTok{(}\DecValTok{100}\NormalTok{, }\DecValTok{0}\NormalTok{, }\DecValTok{100}\NormalTok{)}
\NormalTok{e }\OtherTok{\textless{}{-}} \FunctionTok{runif}\NormalTok{(}\DecValTok{100}\NormalTok{, }\DecValTok{0}\NormalTok{, }\DecValTok{50}\NormalTok{)}
\NormalTok{y }\OtherTok{\textless{}{-}}\NormalTok{ x}\SpecialCharTok{\^{}}\DecValTok{2} \SpecialCharTok{+}\NormalTok{ x }\SpecialCharTok{*}\NormalTok{ e}
\end{Highlighting}
\end{Shaded}

\hypertarget{aufgabe-9.6-rangkorrelation}{%
\subsection{Aufgabe 9.6
(Rangkorrelation)}\label{aufgabe-9.6-rangkorrelation}}

Berechnen Sie in folgenden drei Datensätzen die Pearson-, Kendall- und
Spearman-Korrelationen (r, Rho, Tau) zwischen \(x\) und \(y\). Erklären
Sie die Unterschiede in den Werten. Erstellen Sie zur Hilfe
Streudiagramme der drei Datensätze. Welche Korrelationsmaße sind warum
für welchen Datensatz geeignet oder nicht?

\begin{Shaded}
\begin{Highlighting}[]
\NormalTok{y1 }\OtherTok{\textless{}{-}}\NormalTok{ x }\SpecialCharTok{+}\NormalTok{ e}
\NormalTok{y2 }\OtherTok{\textless{}{-}}\NormalTok{ y1}\SpecialCharTok{\^{}}\DecValTok{4}
\NormalTok{y3 }\OtherTok{\textless{}{-}}\NormalTok{ (x }\SpecialCharTok{{-}} \DecValTok{50}\NormalTok{)}\SpecialCharTok{\^{}}\DecValTok{2} \SpecialCharTok{+} \DecValTok{30} \SpecialCharTok{*}\NormalTok{ e}

\NormalTok{data1 }\OtherTok{\textless{}{-}} \FunctionTok{data.frame}\NormalTok{(}\AttributeTok{x=}\NormalTok{x, }\AttributeTok{y=}\NormalTok{y1)}
\NormalTok{data2 }\OtherTok{\textless{}{-}} \FunctionTok{data.frame}\NormalTok{(}\AttributeTok{x=}\NormalTok{x, }\AttributeTok{y=}\NormalTok{y2)}
\NormalTok{data3 }\OtherTok{\textless{}{-}} \FunctionTok{data.frame}\NormalTok{(}\AttributeTok{x=}\NormalTok{x, }\AttributeTok{y=}\NormalTok{y3)}
\end{Highlighting}
\end{Shaded}


\end{document}
